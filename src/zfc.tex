\chapter{``Zfc'' Standard Extension for Additional Floating-Point Comparison
Instructions, Version 0.1}
\label{sec:zfc}

This chapter describes the Zfc standard extension, which adds instructions for
quiet floating-point comparisons and for the IEEE 754-2019 minimum and maximum
operations.
The Zfc extension depends on the F extension.

\begin{commentary}
The name Zfc and the title's implicit description of minimum and maximum as
comparison instructions are both debatable.
\end{commentary}

\begin{commentary}
An analogous extension compatible with the Zfinx extension could be defined.
\end{commentary}

\section{Minimum and Maximum Instructions}

The FMINI.S and FMAXI.S instructions are defined like the FMIN.S and FMAX.S
instructions, except that if either input is NaN, the result is the
canonical NaN.

If the D extension is implemented, FMINI.D and FMAXI.D instructions are
analogously defined to operate on double-precision numbers.

If the Zfh extension is implemented, FMINI.H and FMAXI.H instructions are
analogously defined to operate on half-precision numbers.

If the Q extension is implemented, FMINI.Q and FMAXI.Q instructions are
analogously defined to operate on quad-precision numbers.

These instructions are encoded like their FMIN and FMAX counterparts, but
with instruction bit 13 set to 1.

\begin{commentary}
These instructions implement the IEEE 754-2019 minimum and maximum operations.
\end{commentary}


\section{Comparison Instructions}

The FLEQ.S and FLTQ.S instructions are defined like the FLE.S and FLT.S
instructions, except that quiet NaN inputs do not cause the invalid
operation exception flag to be set.

If the D extension is implemented, FLEQ.D and FLTQ.D instructions are
analogously defined to operate on double-precision numbers.

If the Zfh extension is implemented, FLEQ.H and FLTQ.H instructions are
analogously defined to operate on half-precision numbers.

If the Q extension is implemented, FLEQ.Q and FLTQ.Q instructions are
analogously defined to operate on quad-precision numbers.

These instructions are encoded like their FLE and FLT counterparts, but
with instruction bit 14 set to 1.

\begin{commentary}
We do not expect analogous comparison instructions will be added to the vector
ISA, since they can be reasonably efficiently emulated using masking.
\end{commentary}
