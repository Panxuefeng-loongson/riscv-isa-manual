\chapter{``Smrnmi'' Standard Extension for Resumable Non-Maskable Interrupts, Version 0.4}
\label{chap:rnmi}

The base machine-level architecture supports only
unresumable non-maskable interrupts (UNMIs), where the NMI jumps to a
handler in machine mode, overwriting the current {\tt mepc} and {\tt mcause}
register values.  If the hart had been executing machine-mode code in
a trap handler, the previous values in {\tt mepc} and {\tt mcause} would not
be recoverable and so execution is not generally resumable.

The Smrnmi extension adds support for resumable non-maskable interrupts
(RNMIs) to RISC-V.  The extension adds four new CSRs ({\tt mnepc},
{\tt mncause}, {\tt mnstatus}, and {\tt mnscratch}) to hold the interrupted state,
and one new instruction, MNRET, to resume from the RNMI handler.

\section{RNMI Interrupt Signals}

The {\tt rnmi} interrupt signals are inputs to
the hart.  These interrupts have higher priority than any other
interrupt or exception on the hart and cannot be disabled by software.
Specifically, they are not disabled by clearing the {\tt mstatus}.MIE
register.

\section{RNMI Handler Addresses}

The RNMI interrupt trap handler address is implementation-defined.

RNMI also has an associated exception trap handler address, which is
implementation defined.

\section{RNMI CSRs}

This proposal adds additional M-mode CSRs to enable a resumable
non-maskable interrupt (RNMI).

\begin{figure*}[h!]
{\footnotesize
\begin{center}
\setlength{\tabcolsep}{4pt}
\begin{tabular}{J}
\instbitrange{MXLEN-1}{0} \\
\hline
\multicolumn{1}{|c|}{\tt mnscratch} \\
\hline
MXLEN \\
\end{tabular}
\end{center}
}
\vspace{-0.1in}
\caption{Resumable NMI scratch register {\tt mnscratch}.}
\label{fig:mnscratch}
\end{figure*}

The {\tt mnscratch} CSR holds an MXLEN-bit read-write register which
enables the NMI trap handler to save and restore the context that was
interrupted.

\begin{figure*}[h!]
{\footnotesize
\begin{center}
\setlength{\tabcolsep}{4pt}
\begin{tabular}{J}
\instbitrange{MXLEN-1}{0} \\
\hline
\multicolumn{1}{|c|}{{\tt mnepc} (\warl)} \\
\hline
MXLEN \\
\end{tabular}
\end{center}
}
\vspace{-0.1in}
\caption{Resumable NMI program counter {\tt mnepc}.}
\label{fig:mnepc}
\end{figure*}

The {\tt mnepc} CSR is an MXLEN-bit read-write register which on entry
to the NMI trap handler holds the PC of the instruction that took the
interrupt.

The low bit of {\tt mnepc} ({\tt mnepc[0]}) is
always zero.  On implementations that support only IALIGN=32, the two low bits
({\tt mnepc[1:0]}) are always zero.

If an implementation allows IALIGN to be either 16 or 32 (by
changing CSR {\tt misa}, for example), then, whenever IALIGN=32, bit
{\tt mnepc[1]} is masked on reads so that it appears to be 0.  This
masking occurs also for the implicit read by the MRET instruction.
Though masked, {\tt mnepc[1]} remains writable when IALIGN=32.

{\tt mnepc} is a \warl\ register that must be able to hold all valid
virtual addresses.  It need not be capable of holding all possible invalid
addresses.
Prior to writing {\tt mnepc}, implementations may convert an invalid address
into some other invalid address that {\tt mnepc} is capable of holding.


\begin{figure*}[h!]
{\footnotesize
\begin{center}
\setlength{\tabcolsep}{4pt}
\begin{tabular}{cU}
\instbit{MXLEN-1} &
\instbitrange{MXLEN-2}{0} \\
\hline
\multicolumn{1}{|c|}{1} &
\multicolumn{1}{c|}{NMI Cause (\warl)} \\
\hline
1 & MXLEN-1 \\
\end{tabular}
\end{center}
}
\vspace{-0.1in}
\caption{Resumable NMI cause {\tt mncause}.}
\label{fig:mncause}
\end{figure*}

The {\tt mncause} CSR holds the reason for the NMI, with bit MXLEN-1 set to
1, and the NMI cause encoded in the least-significant bits or zero if
NMI causes are not supported.

\begin{figure*}[h!]
{\footnotesize
\begin{center}
\setlength{\tabcolsep}{4pt}
\begin{tabular}{TRFcFcF}
\instbitrange{MXLEN-1}{13} &
\instbitrange{12}{11} &
\instbitrange{10}{8} &
\instbit{7} &
\instbitrange{6}{4} &
\instbit{3} &
\instbitrange{2}{0} \\
\hline
\multicolumn{1}{|c|}{\em Reserved} &
\multicolumn{1}{c|}{MNPP (\warl)} &
\multicolumn{1}{c|}{\em Reserved} &
\multicolumn{1}{c|}{MNPV (\warl)} &
\multicolumn{1}{c|}{\em Reserved} &
\multicolumn{1}{c|}{NMIE} &
\multicolumn{1}{c|}{\em Reserved} \\
\hline
MXLEN-13 & 2 & 3 & 1 & 3 & 1 & 3 \\
\end{tabular}
\end{center}
}
\vspace{-0.1in}
\caption{Resumable NMI status register {\tt mnstatus}.}
\label{fig:mnstatus}
\end{figure*}

The {\tt mnstatus} CSR holds a two-bit field, MNPP, which on entry to the trap
handler holds the privilege mode of the interrupted context, encoded
in the same manner as {\tt mstatus}.MPP.
It also holds a one-bit field, MNPV, which on entry to the trap handler holds
the virtualization mode of the interrupted context, encoded in the same
manner as {\tt mstatus}.MPV.

{\tt mnstatus} also holds the NMIE bit.
When NMIE=1, nonmaskable interrupts are enabled.
When NMIE=0, {\em all} interrupts are disabled.

When NMIE=0, the hart behaves as though {\tt mstatus}.MPRV were clear,
regardless of the current setting of {\tt mstatus}.MPRV.

Attempts to write the NMIE bit via the {\tt mnstatus} CSR have no effect.

\begin{commentary}
We do not allow software to clear the NMIE bit, because otherwise RNMIs
would not truly be non-maskable.

Previous versions of this specification allowed software to set the NMIE bit,
so that RNMIs could be taken during the handling of other RNMIs.
However, this scheme does not suffice to support software nesting of RNMIs.
To support software nesting, the nested handler's return path would need to
clear the NMIE bit, which we wish to disallow.
Since this feature is otherwise of limited use, we have removed it.

Software that wishes to minimize the latency until the next RNMI is taken can
follow the top-half/bottom-half model, where the RNMI handler itself only
enqueues a task to a task queue then returns.
The bulk of the interrupt servicing is performed later, with RNMIs enabled.

It would have sufficed not to expose the read-only NMIE bit to software at
all, but making it software-visible aids debugging.
\end{commentary}

The other
bits in {\tt mnstatus} are {\em reserved}; software should write zeros and
hardware implementations should return zeros.

\section{MNRET Instruction}

MNRET is an M-mode-only instruction that uses the values in {\tt mnepc} and
{\tt mnstatus} to return to the program counter, privilege mode,
and virtualization mode of the interrupted context.
This instruction also sets {\tt mnstatus}.NMIE.

\section{RNMI Operation}

When an RNMI interrupt is detected, the interrupted PC is written to
the {\tt mnepc} CSR, the type of RNMI to the {\tt mncause} CSR, and the
privilege mode of the interrupted context to the {\tt mnstatus} CSR.
The {\tt mnstatus}.NMIE bit is cleared, masking all interrupts.

The hart then enters machine-mode and jumps to the RNMI trap handler
address.

The RNMI handler can resume original execution using the new MNRET
instruction, which restores the PC from {\tt mnepc}, the privilege mode
from {\tt mnstatus}, and also sets {\tt mnstatus}.NMIE, which
re-enables interrupts.

If the hart encounters an exception while {\tt mnstatus}.NMIE bit is clear, the
exception state is written to {\tt mepc} and {\tt mcause}, {\tt mstatus}.MPP is
set to M-mode, and the hart jumps to the RNMI exception handler
address.

\begin{commentary}
Traps in the RNMI handler can only be resumed if they occur while
the handler was servicing an interrupt that occurred outside of
machine-mode.
\end{commentary}

\begin{commentary}
The Smrnmi extension does not change the behavior of the MRET and SRET
instructions.
In particular, MRET and SRET are unaffected by the {\tt mnstatus}.NMIE bit,
and their execution does not alter the {\tt mnstatus}.NMIE bit.
\end{commentary}
